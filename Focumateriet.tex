

\renewcommand{\dateseparator}{-} %Omdefinierar hur datum visas

\renewcommand{\forening}{Focumateriet}

\begin{center}
\LARGE{\textbf{Arbetsordning \forening}}
\end{center}
%\vspace{1cm}



\section{Allmänt}
\begin{itemize}

\item \forening \  är sektionens Focumateri.
\item \forening \ har kommittéstatus på sektionen.
\item \forening: syfte är att sköta automater och flipperspel som ägs av sektionen och befinner sig i sektionslokalen eller i direkt närhet till sektionslokalen.
\end{itemize}

\forening består av följande poster:
\begin{itemize}
\item Ordförande, Kaptenen
\item Vice ordförande, Automatpirat
\item Kassör, Kistväktaren
\item 0--5 övriga ledamöter
\end{itemize}

Där Ordförande, Vice ordförande och Kassör är förtroendeposter. 

De övriga ledamöterna utses i nära anslutning till val av förtroendeposter och tjänstgör
under ett års tid. De ska genomgå ett samtal med valberedningen innan de kan rekommenderas för tjänstgöring. Övriga medlemmar skall
godkännas av sektionsstyrelsen. Om någon av de övriga
medlemmarna ej tillsätts enligt ovan kan de fyllnadsväljas av sektionsstyrelsen
i samråd med sittande förtroendevalda.

\section{Generella åligganden}

Det åligger varje sektionskommitté:
\begin{description}
    \item[att] på det ordinarie sektionsmöte som följer på det, då kommittén blivit invald, presentera en verksamhetsplan för det kommande verksamhetsåret.
      \item[att] tillsammans med Fysikteknologsektionens valberedning lägga fram förslag på efterträdare till respektive sektionskommitté.
      \item[att] inför inval av poster och förtroendeposter i respektive sektionskommitté, tillse att samtliga sektionens medlemmar har givits fullgod möjlighet att med lätthet införskaffa information och bekantskap med respektive sektionskommitté och dess arbetsuppgifter.
      \item[att] kontinuerligt dokumentera sin verksamhet, som stöd till näst\-komm\-ande års kommitté.
      \item[att] delta i de av Djungelpatrullen anordnade städdagarna två gånger per
      läsår. Om detta uppfylls får de närvarande gå gratis på nästa
      sektionsaktivafest.
      \item[att] under inga omständigheter vid med sektionen associerat arrangemang eller vid med sektionen associerad festlighet servera, försälja eller bjuda på öhl av märket Pripps, förutom vid häfv eller arrangemang i kårhuset.
      
    \end{description}

\section{Specifika åligganden:}

Det åligger \forening :
\begin{description}
\item[att] handha Focumaten, samt, efter sektionsstyrelsens bestämmande, av sek-
tionen ägda automater samt av sektionen ägd elektronisk utrust-
ning.

\item[att] vid sektionstillställningar, eller vid sektionstillställning jämförbar fest, bära haklapp och spypåse samt ha städdon i omedelbar närhet för att få närvara. F6 bedömmer enväldigt och utan chans till överklagan, huruvida förutsättningarna enligt ovan är uppfyllda.
 
\item[att] se till att teknologerna har tillgång till kaffe, te och socker i sektionslokalen till självkostnadspris.

\item[att] inför sektionsmötet lp4, med styrets godkännande, bestämma den hatt som nästkommande talmanspresidie ska använda.

\end{description}

\subsection{Valberedning}
\subsubsection{Förberedelse}
Inför valberedningen är kommittén skyldig att bilda sig en detaljerad uppfattning om varje aspirant till föreningen som skall valberedas.\\
Därtill och kommittén skyldig att i samråd med valberedningen upprätta en kravprofil för kommittén i helhet samt för kommitténs förtroendeposter.

\subsubsection{Intervjuer}
Under valberedningens intervjuer skall \forening \ utse två representanter att deltaga under dessa samt under de efterföljande samtalen i syfte att nominera kandidater.\\
De representanter som utses får själv inte valberedas för eller söka någon post i denna förening.

\section{Postspecifika uppdrag}

\subsection{Ordförande, Kaptenen}
\forening s ordförande är tillsammans med \forening s kassör ansvarig för kommitténs ekonomi. Ordförande skall leda \forening s arbete samt vara en kontaktlänk mellan \forening \ och övriga kommittéer, nämnder samt sektionsstyrelsen. 

\begin{itemize}
\item \forening s ordförande är ledamot i sektionsstyrelsen.
\end{itemize}

Det åligger \forening s ordförande:
\begin{description}
\item[att] tillse att kommitténs åligganden utförs.
\item[att] leda focumateriets arbete. 
\item[att] i kassörens frånvaro sköta Focumateriets ekonomi. 
\end{description}


\subsection{Vice ordförande, Automatpiraten}
Det åligger \forening s vice ordförande, Styrmannen:
\begin{description}
\item[att] vara kommitténs suppleant i sektionsstyrelsen
\item[att] bistå ordföranden i dennes uppgifter så att de utförs på bästa sätt.
\item[att] i ordförandens frånvaro utföra dennes uppdrag.
\item[att] ansvara för sektionens automater.
\end{description}


\subsection{Kassör}
Kassören är tillsammans med ordföranden ansvarig för \forening \ ekonomi. Det åligger kassören att kontinuerligt föra en granskningsbar redovisning gällande \forening \ ekonomi.\\

\subsubsection{Det åligger kassören i varje sektionskommitté:}
\begin{description}
\item[att] mot revisorerna kontinuerligt redovisa för den ekonomiska situationen.
\end{description}


\subsection{Övriga ledamöter}
\forening s övriga ledamöter skall hjälpa \forening s ordförande i dennes uppgifter så att de utförs på bästa sätt.

\section{Tolkning och ändring}
Tolkning av denna arbetsordning görs av sektionsstyrelsen.\\ Då Fysikteknologsektionens Stadga, Reglemente eller policyer motsäger denna arbetsordning har Stadga och Reglemente företräde. Då kårens Stadga, Reglemente eller policyer motsäger denna arbetsordning har kårens Stadga, Reglemente och policyer företräde.\\
Ändring och tillägg av denna arbetsordning görs av sektionsstyrelsen. Ändring skall redovisas vid nästkommande sektionsmöte. 

Första versionen av denna arbetsordning antogs av sektionsmötet läsperiod 1 verksamhetsåret 14/15.

\newpage
