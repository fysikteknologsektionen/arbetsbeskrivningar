\renewcommand{\dateseparator}{-} %Omdefinierar hur datum visas

\renewcommand{\forening}{F6 }

\begin{center}
\LARGE{\textbf{Arbetsordning F6}}
\end{center}
%\vspace{1cm}


\section{Allmänt}
\begin{itemize}

\item \forening är sektionens sexmästeri. 

\item \forening har kommittéstatus på sektionen.

\item \forening :s syfte är att ansvara för kalas- och tentamensfestlighet på sektionen. \forening ombesörjer att festligheterna planeras och genomförs på bästa sätt och med hög standard. 
\end{itemize}
\forening består av följande poster:
\begin{itemize}
\item Ordförande, Sexmästaren
\item Vice ordförande, Sexreteraren
\item Kassör
\item 0-6 ledamöter
\end{itemize}

De övriga ledamöterna utses i nära anslutning till val av förtroendeposter och tjänstgör under ett års tid. De ska genomgå ett samtal med valberedningen innan de kan rekommenderas för tjänstgöring. Övriga medlemmar skall godkännas av sektionsstyrelsen. Om någon av de övriga medlemmarna ej tillsätts enligt ovan kan de fyllnadsväljas av sektionstyrelsen. 
i samråd med sittande förtroendevalda.

\section{Generella åligganden}

Det åligger varje sektionskommitté:
\begin{description}
    \item[att] på det ordinarie sektionsmöte som följer på det, då kommittén blivit invald, presentera en verksamhetsplan för det kommande verksamhetsåret.
      \item[att] tillsammans med Fysikteknologsektionens valberedning lägga fram förslag på efterträdare till respektive sektionskommitté.
      \item[att] inför inval av poster och förtroendeposter i respektive sektionskommitté, tillse att samtliga sektionens medlemmar har givits fullgod möjlighet att med lätthet införskaffa information och bekantskap med respektive sektionskommitté och dess arbetsuppgifter.
      \item[att] kontinuerligt dokumentera sin verksamhet, som stöd till näst\-komm\-ande års kommitté.
      \item[att] delta i de av Djungelpatrullen anordnade städdagarna två gånger per
      läsår. Om detta uppfylls får de närvarande gå gratis på nästa
      sektionsaktivafest.
      \item[att] under inga omständigheter vid med sektionen associerat arrangemang eller vid med sektionen associerad festlighet servera, försälja eller bjuda på öhl av märket Pripps, förutom vid häfv eller arrangemang i kårhuset.
      
    \end{description}


\section{Specifika åligganden}
Det åligger \forening:

\subsection{Allmänt}
\begin{description}
\item[att] Sexmästaren och en ledamot går SUS-utbildning så att någon av dem kan vara serveringsansvarig vid de arrangemang som kräver detta.

\end{description}

\subsection{Poster och representanter}
\begin{description}
\item[att] Inför nollbalen utse en representant bland sina medlemmar att vara Balnågonting behjälplig inför och under arrangerandet av balen.
\end{description}


\subsection{Festligher}
\begin{description}
\item[att] Annordna en gasque minst en gång per läsperiod.

\item[att] ansvara för kalas- och tentamensfestlighetsverksamhet på sektionen. 

\end{description}


\subsection{Övrigt}
\begin{description}
\item[att] I samråd med FnollK och Djungelpatrullen tillse att sektionsmärket i Olgas trappor målas i samband med mottagningen.

\item[att] Under mottagningen vara ett kompliment till FnollK.
\end{description}

\subsection{Valberedning}
\subsubsection{Förberedelse}
Inför valberedningen är kommittén skyldig att bilda sig en detaljerad uppfattning om varje aspirant till föreningen som skall valberedas.\\
Därtill och kommittén skyldig att i samråd med valberedningen upprätta en kravprofil för kommittén i helhet samt för kommitténs förtroendeposter.

\subsubsection{Intervjuer}
Under valberedningens intervjuer skall kommittén F6utse två representanter att deltaga under dessa samt under de efterföljande samtalen i syfte att nominera kandidater.\\
De representanter som utses får själv inte valberedas för eller söka någon post i denna förening.



\section{Postspecifika uppdrag}

\subsection{Ordförande, Sexmästaren}
\forening ordförande är tillsammans med F6 kassör ansvarig för kommitténs ekonomi. Ordförande skall leda \forening arbete samt vara en kontaktlänk mellan \forening och övriga kommittéer, nämnder samt sektionsstyrelsen. 



\begin{itemize}

\item \forening ordförande är ledamot i sektionsstyrelsen.

\item \forening ordförande är sektionens representant i Gasquerådet om \forening beslutar att vara medlemmar i Gasquerådet.

\end{itemize}


Det åligger \forening :s ordförande, Sexmästaren:
\begin{description}

\item[att] Leda \forening arbete

\item[att] Tillse att kommitténs åligganden utförs.

\item[att] Efter årets slut, i samråd med kassören, sammanställa en verksamhetsberättelse. Verksamhetsberättelsen skall vara sammanställd och inskickad till sektionsstyrelsen senast 10 dagar innan första sektionsmötet efter verksamhetsårets slut.

\end{description}

\subsection{vice Ordförande, Sexreteraren}
Det åligger \forening vice ordförande, Sexreteraren:
\begin{description}
\item[att] I ordförandens frånvaro utföra dennes uppdrag.

\item[att] Föra protokoll vid \forening möten.

\item[att] Vara kommitténs suppleant i sektionsstyrelsen

\end{description}



\subsection{Kassör}
Kassören är tillsammans med ordföranden ansvarig för F6 ekonomi.\\ 
Det åligger \forening kassör:
\begin{description}
\item[att] Kontinuerligt föra en granskningsbar redovisning gällande \forening ekonomi.
\end{description}

\subsubsection{Det åligger kassören i varje sektionskommitté att:}
\begin{description}
\item[att] mot revisorerna kontinuerligt redovisa för den ekonomiska situationen.

\end{description}

\subsection{Öhl- och spritchef}
Det åligger \forening att utse en ledamot till Öhl- och spritchef.\\
Det åligger \forening Öhl- och spritchef:
\begin{description}
\item[att] Ansvara för inköp samt hantering av kommitténs alkohol under arrangemang.
\end{description}

\subsection{Övriga ledamöter}
F6 övriga ledamöter skall vara de förtroendevalda i F6 behjälpliga i verksamheten. Dessa skall dessutom utefter bästa förmåga hjälpa sexmästaren att upprätthålla en god stämning, både inom kommittén och på sina arrangemang.

\section{Tolkning och ändring}
Tolkning av denna arbetsordning görs av sektionsstyrelsen.\\ Då Fysikteknologsektionens Stadga, Reglemente eller policyer motsäger denna arbetsordning har Stadga och Reglemente företräde. Då kårens Stadga, Reglemente eller policyer motsäger denna arbetsordning har kårens Stadga, Reglemente och policyer företräde.\\
Ändring och tillägg av denna arbetsordning görs av sektionsstyrelsen. Ändring skall redovisas vid nästkommande sektionsmöte. 

Första versionen av denna arbetsordning antogs av sektionsmötet läsperiod 1 verksamhetsåret 14/15.

\newpage