

\renewcommand{\dateseparator}{-} %Omdefinierar hur datum visas

\renewcommand{\forening}{FnollK}

\begin{center}
\LARGE{\textbf{Arbetsordning \forening}}
\end{center}
%\vspace{1cm}

\section{Allmänt}
\begin{itemize}
\item \forening \ är sektionens mottagningskommitté.

\item \forening \ har kommittéstatus på sektionen.

\item \forening:s syfte är att genomföra en värdig mottagning i enighet med kårens och sektionens intentioner. 
\end{itemize}


\forening består av förljande förtroendeposter:
\begin{itemize}
\item Ordförande
\item Vice ordförande
\item Kassör
\item 2--4 ledamöter.
\end{itemize}

\forening \ har noll övriga medlemmar.


\section{Generella åligganden}

Det åligger varje sektionskommitté:
\begin{description}
    \item[att] på det ordinarie sektionsmöte som följer på det, då kommittén blivit invald, presentera en verksamhetsplan för det kommande verksamhetsåret.
      \item[att] tillsammans med Fysikteknologsektionens valberedning lägga fram förslag på efterträdare till respektive sektionskommitté.
      \item[att] inför inval av poster och förtroendeposter i respektive sektionskommitté, tillse att samtliga sektionens medlemmar har givits fullgod möjlighet att med lätthet införskaffa information och bekantskap med respektive sektionskommitté och dess arbetsuppgifter.
      \item[att] kontinuerligt dokumentera sin verksamhet, som stöd till näst\-komm\-ande års kommitté.
      \item[att] delta i de av Djungelpatrullen anordnade städdagarna två gånger per
      läsår. Om detta uppfylls får de närvarande gå gratis på nästa
      sektionsaktivafest.
      \item[att] under inga omständigheter vid med sektionen associerat arrangemang eller vid med sektionen associerad festlighet servera, försälja eller bjuda på öhl av märket Pripps, förutom vid häfv eller arrangemang i kårhuset.
      
    \end{description}

\section{Specifika åligganden}
Det åligger \forening :

\subsection{Allmänt}
\begin{description}



\item[att] arrangera aktiviteter för F- och TM-nollan som syftar till att in-
förliva Nollan i livet som F- respektive TM-teknolog både vad
gäller studier och det studiesociala livet.

\item[att] främja kamratskapen mellan nollan samt, om gamble så tycker, mellan nollan och gamble och mellan nollan och nymble, samt mellan nollan och valen Åke om valen Åke så tycker.

\end{description}
\subsection{Medlemmar och representanter}
\begin{description}

\item[att] bland sina medlemmar utse en Phadderchef

\item[att] inför nollbalen utse en representant utav sina medlemmar att vara Balnågonting behjälplig inför och under arrangerandet av balen.

\end{description}

\subsection{Mottagningen}
\begin{description}
\item[att] under ledning av \forening s ordförande planera och leda mottagningens aktiviteter i samråd med berörda organ.

\item[att] tillse att det under mottagningen skall det finnas tillräcklig mängd aktiviteter att fylla nollans fritid.

\item[att] ordna nolluppdrag

\item[att] framställa F-nollmodulen

\item[att] utvärdera mottagningen.

\item[att] tillse att alla Nollan, om de
så önskar bär nollbricka.

\item[att] söka sponsring för mottagningen.

\item[att] tillse att valen Åke bär nollbricka om gamble så tycker.

\end{description}

\subsection{Övrigt}
\begin{description}
\item[att] i samråd med Djungelpatrullen och F6 tillse att sektionsmärket i Olgas trappor målas i samband med mottagningen.

\item[att] ordna en trevlig sammankomst med andra nollningskommittéer


\end{description}

\subsection{Valberedning}
\subsubsection{Förberedelse}
Inför valberedningen är kommittén skyldig att bilda sig en detaljerad uppfattning om varje aspirant till föreningen som skall valberedas.\\
Därtill och kommittén skyldig att i samråd med valberedningen upprätta en kravprofil för kommittén i helhet samt för kommitténs förtroendeposter.

\subsubsection{Intervjuer}
Under valberedningens intervjuer skall \forening \ utse två representanter att deltaga under dessa samt under de efterföljande samtalen i syfte att nominera kandidater.\\
De representanter som utses får själv inte valberedas för eller söka någon post i denna förening.

\section{Postspecifika uppdrag}

\subsection{Ordförande}
\forening s ordförande är tillsammans med \forening s kassör ansvarig för kommitténs ekonomi. Ordförande skall leda \forening s arbete samt vara en kontaktlänk mellan \forening \ och övriga kommittéer, nämnder samt sektionsstyrelsen. 

\begin{itemize}

\item \forening s ordförande är ledamot i sektionsstyrelsen.

\item \forening s ordförande skall vara sektionens representant i Chalmers studentkårs sammarbetsorgan för mottagningen, MOS.
\end{itemize}


Det åligger \forening s ordförande:
\begin{description}

\item[att] tillse att kommitténs åligganden utförs.

\item[att] ansvara för att god kontakt hålls med andra nollningskommittéer.

\end{description}

\subsection{vice Ordförande}
Det åligger \forening s vice ordförande:
\begin{description}

\item[att] vara kommitténs suppleant i sektionsstyrelsen

\item[att] bistå ordföranden i dennes uppgifter så att de utförs på bästa sätt.
\item[att] i ordförandens frånvara utföra dennes uppdrag.

\end{description}



\subsection{Kassör}
\subsubsection{Allmänt}
Kassören är tillsammans med ordföranden ansvarig för \forening s ekonomi. Det åligger kassören att kontinuerligt föra en granskningsbar redovisning gällande \forening s ekonomi.

\subsubsection{Åligganden}
Det åligger kassören i varje sektionskommitté:
\begin{description}
\item[att] mot revisorerna kontinuerligt redovisa för den ekonomiska situationen.

\end{description}

\subsection{Phadderchef}
Det åligger \forening s Phadderchef att tillse att nollans behov av phaddrar tillgodoses.\\
Phaddergruppschefen skall därtill nominera ett erfoderligt antal phaddergruppsansvariga till sektionsstyrelsen.


\subsection{Övriga ledamöter}
\forening s övriga ledamöter skall hjälpa \forening s ordförande i dennes uppgifter så att de utförs på bästa sätt.

\section{Tolkning och ändring}
Tolkning av denna arbetsordning görs av sektionsstyrelsen.\\ Då Fysikteknologsektionens Stadga, Reglemente eller policyer motsäger denna arbetsordning har Stadga och Reglemente företräde. Då kårens Stadga, Reglemente eller policyer motsäger denna arbetsordning har kårens Stadga, Reglemente och policyer företräde.\\
Ändring och tillägg av denna arbetsordning görs av sektionsstyrelsen. Ändring skall redovisas vid nästkommande sektionsmöte. 

Första versionen av denna arbetsordning antogs av sektionsmötet läsperiod 1 verksamhetsåret 14/15.

\newpage
