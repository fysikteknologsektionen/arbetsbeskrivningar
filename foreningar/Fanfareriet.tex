\documentclass[a4paper]{article}
\usepackage{preamble}
\begin{document}

\renewcommand{\forening}{Fanfareriet} % Används för headern bl.a.

\begin{foreningenv}{\forening{}} % Hanterar mycket av logiken kring enskild/gemensam kompilering. Argumentet blir rubriken
    \section{Allmänt}
    \begin{itemize}
        \item Fanfareriet har funktionärsstatus på sektionen.
        \item Fanfareriet består av en flaggmarskalk och 1-2 fanbärare.
        \item Fanfareriets syfte är att ta hand om sektionens fanor och flaggor.
        \item Dragos är Fanfareriets högsta befäl.
    \end{itemize}
    
    \section{Generella åligganden}
    \aliggsektfunkt{}
    
    \section{Specifika åligganden}
    Det åligger \forening{}:
    \begin{description}
        \item[att] sköta sektionens fanor och flaggor.
    \end{description}
    
    \section{Postspecifika åligganden}
    \subsection{Flaggmarskalken}
    Det åligger flaggmarskalken:
    \begin{description}
        \item[att] hissa Fanstyget i sektionens flaggstång på:
        \begin{itemize}
            \item dag då sektionsmöte hålles.
            %\item dag då Nollkamp utkämpas. nytt
            \item Nollans uppropsdag.
            %\item Sektionens dag. nytt
            \item sektionens middag. %nytt
            \item Emil och Emilias namnsdag den 14 november.
            \item Lee Falks dödsdag den 13 mars (halv stång).
            \item annan dag av synnerlig vikt för sektionen.
        \end{itemize}
        \item[att] hissa svensk flagga i sektionens flaggstång på allmän flaggdag, annan dag av synnerlig vikt för Chalmers eller nationen samt att härvid tillse att officiella flaggningsregler i största möjliga mån tillämpas.
        \item[att] vid tillfällen enligt eget gott omdöme, hissa sektionens eventuella övriga flaggor i sektionens flaggstång.
        \item[att] %vid ovannämnda tillfällen
        enligt eget gott omdöme %nytt
        informera sektionens medlemmar om anledningen till att Fanstyget, svensk flagga eller annan flagga är hissad.
        \item[att] meddela styrelsen om skador eller andra brister hos Fanstyget, sektionens svenska flagga, sektionens flaggstång eller sektionens eventuella övriga flaggor.
        \item[att] vid förfall utse ersättare att utföra flaggmarskalkens åligganden.
    \end{description}
    
    \subsection{Fanbärarna}
    Det åligger Fanbärarna:
    \begin{description}
        \item[att] när så krävs bära Fysikteknologsektionens fana, iklädd högtidsdräkt, vid:
        \begin{itemize}
    	    \item Valborg.
    	    \item Mottagningens första dag.
    	    \item Mösspåtagningen 1 oktober.
    	    \item Andra tillfällen av vikt när sektionens fana ska bäras.
    	\end{itemize}
    \end{description}
\end{foreningenv}

\end{document}