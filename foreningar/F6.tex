\documentclass[a4paper]{article}
\usepackage{preamble}
\begin{document}

\renewcommand{\forening}{F6} % Används för headern bl.a.

\begin{foreningenv}{\forening{}} % Hanterar mycket av logiken kring enskild/gemensam kompilering. Argumentet blir rubriken
    \section{Allmänt}
    \begin{itemize}
        \item F6 är sektionens sexmästeri. 
        \item F6 har kommittéstatus på sektionen.
        \item F6:s syfte är att ansvara för kalas- och tentamensfestlighet på sektionen. F6 ombesörjer att festligheterna planeras och genomförs på bästa sätt och med hög standard. 
    \end{itemize}
    F6 består av följande poster:
    \begin{itemize}
        \item Ordförande, Sexmästaren
        \item Vice ordförande, Sexreteraren
        \item Kassör
        \item 0-6 ledamöter
    \end{itemize}
    De övriga ledamöterna utses i nära anslutning till val av förtroendeposter och tjänstgör under ett års tid. De ska genomgå ett samtal med valberedningen innan de kan rekommenderas för tjänstgöring. Övriga ledamöter skall godkännas av sektionsstyrelsen. Om någon av de övriga ledamöterna ej tillsätts enligt ovan kan de fyllnadsväljas av sektionstyrelsen i samråd med sittande förtroendevalda.
    
    \section{Generella åligganden}
    \aliggkom{}
    
    \section{Specifika åligganden}
    Det åligger F6:
    
    \subsection{Allmänt}
    \begin{description}
        \item[att] Sexmästaren och en ledamot går SUS-utbildning så att någon av dem kan vara serveringsansvarig vid de arrangemang som kräver detta.
    \end{description}
    
    \subsection{Poster och representanter}
    \begin{description}
        \item[att] inför Nollbalen utse en representant bland sina medlemmar att vara Balnågonting behjälplig inför och under arrangerandet av balen.
    \end{description}
    
    \subsection{Festligher}
    \begin{description}
        \item[att] annordna en gasque minst en gång per läsperiod.
        \item[att] ansvara för kalas- och tentamensfestlighetsverksamhet på sektionen.
    \end{description}
    
    \subsection{Övrigt}
    \begin{description}
        \item[att] i samråd med FnollK och Djungelpatrullen tillse att sektionsmärket i Olgas trappor målas i samband med mottagningen.
        \item[att] under mottagningen vara ett komplement till FnollK.
    \end{description}
    
    \subsection{Valberedning}
    \aliggvalber{}
    
    \section{Postspecifika uppdrag}
    \subsection{Ordförande, Sexmästaren}
    F6:s ordförande är tillsammans med F6:s kassör ansvarig för kommitténs ekonomi. Ordförande skall leda F6:s arbete samt vara en kontaktlänk mellan F6 och övriga kommittéer, nämnder samt sektionsstyrelsen. 
    
    \begin{itemize}
        \item F6:s ordförande är ledamot i sektionsstyrelsen.
        \item F6:s ordförande är sektionens representant i Gasquerådet om F6 beslutar att vara medlemmar i Gasquerådet.
    \end{itemize}
    
    Det åligger F6:s ordförande, Sexmästaren:
    \begin{description}
        \item[att] leda F6:s arbete och tillse att deras åligganden utförs.
        \item[att] efter årets slut, i samråd med kassören, sammanställa en verksamhetsberättelse. Verksamhetsberättelsen skall vara sammanställd och inskickad till sektionsstyrelsen senast 10 dagar innan första sektionsmötet efter verksamhetsårets slut.
    \end{description}
    
    \subsection{Vice ordförande, Sexreteraren}
    Det åligger F6:s vice ordförande, Sexreteraren:
    \begin{description}
        \item[att] i ordförandens frånvaro utföra dennes uppdrag.
        \item[att] föra protokoll vid F6:s möten.
        \item[att] vara kommitténs suppleant i sektionsstyrelsen.
    \end{description}
    
    \subsection{Kassör}
    Kassören är tillsammans med ordföranden ansvarig för F6:s ekonomi.
    
    \subsubsection{Det åligger kassören i varje sektionskommitté att:}
    \begin{description}
        \item[att] kontinuerligt föra en granskningsbar redovisning gällande kommitténs ekonomi.
        \item[att] mot revisorerna kontinuerligt redovisa för den ekonomiska situationen.
    \end{description}
    
    \subsection{Öhl- och spritchef}
    Det åligger F6 att utse en ledamot till Öhl- och spritchef.
    
    Det åligger F6:s Öhl- och spritchef:
    \begin{description}
        \item[att] ansvara för inköp samt hantering av kommitténs alkohol under arrangemang.
    \end{description}
    
    \subsection{Övriga ledamöter}
    F6:s övriga ledamöter skall vara de förtroendevalda i F6 behjälpliga i verksamheten. Dessa skall dessutom utefter bästa förmåga hjälpa Sexmästaren att upprätthålla en god stämning, både inom kommittén och på sina arrangemang.
\end{foreningenv}

\end{document}