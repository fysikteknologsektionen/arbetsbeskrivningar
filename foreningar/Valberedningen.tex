\documentclass[a4paper]{article}
\usepackage{preamble}
\begin{document}

%%nyttnytt: bytt itemize->description
\renewcommand{\forening}{Valberedningen} % Används för headern bl.a.

\begin{foreningenv}{\forening{}} % Hanterar mycket av logiken kring enskild/gemensam kompilering. Argumentet blir rubriken
    \section{Uppdrag}
    Valberedningen har i uppdrag av sektionsmötet att till sektionsmöten med inval presentera lämpliga kandidater till utvalda poster.
    
    \section{Allmänt}
    \begin{itemize}
        \item Valberedningen består av 3-7 ledamöter.
        \item Valberedningen skall ej vara medlem i kommitté eller nämnd som har representant i sektionsstyrelsen, samt heller inte vara medlem i sektionsstyrelsen.
        \item Vid valberedningens första möte väljs internt ordförande och vice ordförande. Mötet är beslutsfört om minst tre ledamöter är närvarande.
    \end{itemize}
    
    \section{Generella åligganden}
    \aliggsektfunkt{}
    
    \section{Specifika åligganden}
    Det åligger \forening{}:
    \begin{description} 
        \item[att] vid sitt första sammanträde internt välja ordförande och vice ordförande.
        \item[att] valbereda samtliga poster i sektionsstyrelsen.
        % \item[att] valbereda revisorer. Togs bort 3 okt
        \item[att] valbereda övriga förtroendeposter enligt reglementet.
        \item[att] valbereda samtliga personer som vill valberedas till poster där detta krävs.
        \item[att] anslå nomineringar minst sju veckodagar före sektionsmötet då invalet sker.
        \item[att] tillsammans med berörd styrelse, kommitté, studienämnd eller funktionär ta fram kravprofil för posten som valbereds. 
        \item[att] tillsammans med berörd styrelse, kommitté, studienämnd eller funktionär tillse att aspningen lyfter fram  egenskaper enligt kravprofilen.
        \item[att] i frånvaro av lämpliga kandidater, i mån av tid, söka reda på sådana.
        \item[att] kontinuerligt dokumentera arbetsgångar och liknande för att behålla kontinuitet inför efterföljande år.
    \end{description}
    
    \section{Valprocessen}
    \begin{itemize}
        \item Valberedningen är beslutsför då ordförande, minst två ytterligare ledamöter samt minst en representant från berörd styrelse, kommitté, studienämnd eller funktionär är närvarande.
        \item Då representant från berörd kommitté, studienämnd eller funktionär saknas får ledamot i styrelsen ersätta denna.
        \item Vid intervjuprocess skall 2-3 ledamöter ur valberedningen, varav minst en av ordförande eller vice ordförande, samt 1-2 ledamöter från berörd styrelse, kommitté, studienämnd eller funktionär vara närvarande.
        \item Valberedning skall i sitt arbete ej vara jävig.
    \end{itemize}
    
    \section{Postspecifika uppdrag}
    \subsection{Ordförande}
    Det åligger valberedningens ordförande:
    \begin{description}
        \item[att] vara sammankallande för valberedningen.
        \item[att] vid lika röstresultat fälla avgörande röst.
    \end{description}
    
    \subsection{Vice ordförande}
    Det åligger valberedningens vice ordförande:
    \begin{description}
        \item[att] i ordförandes frånvara anta dennes åligganden.
        \item[att] vara ordföranden behjälplig.
    \end{description}
\end{foreningenv}

\end{document}