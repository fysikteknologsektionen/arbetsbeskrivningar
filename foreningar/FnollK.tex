\documentclass[a4paper]{article}
\usepackage{preamble}
\begin{document}

\renewcommand{\forening}{FnollK} % Används för headern bl.a.

\begin{foreningenv}{\forening{}} % Hanterar mycket av logiken kring enskild/gemensam kompilering. Argumentet blir rubriken
    \section{Allmänt}
    \begin{itemize}
        \item FnollK är sektionens mottagningskommitté.
        \item FnollK har kommittéstatus på sektionen.
        \item FnollK:s syfte är att genomföra en värdig mottagning i enlighet med kårens och sektionens intentioner. 
    \end{itemize}
    
    FnollK består av följande förtroendeposter:
    \begin{itemize}
        \item Ordförande
        \item Vice ordförande
        \item Kassör
        \item 2--4 ledamöter
    \end{itemize}
    
    \section{Generella åligganden}
    \aliggkom{}
    
    \section{Specifika åligganden}
    Det åligger FnollK:
    
    \subsection{Allmänt}
    \begin{description}
        \item[att] arrangera aktiviteter för F- och TM-Nollan som syftar till att införliva Nollan i livet som F- respektive TM-teknolog både vad gäller studier och det studiesociala livet.
        \item[att] främja kamratskapen mellan Nollan samt, om gamble så tycker, mellan Nollan och gamble och mellan Nollan och nymble, samt mellan Nollan och valen Åke om valen Åke så tycker.
    \end{description}
    
    \subsection{Medlemmar och representanter}
    \begin{description}
        \item[att] bland sina medlemmar utse en Phadderchef.
        \item[att] inför Nollbalen utse en representant utav sina medlemmar att vara Balnågonting behjälplig inför och under arrangerandet av balen.
    \end{description}
    
    \subsection{Mottagningen}
    \begin{description}
        \item[att] under ledning av FnollK:s ordförande planera och leda mottagningens aktiviteter i samråd med berörda organ.
        \item[att] tillse att det under mottagningen skall det finnas tillräcklig mängd aktiviteter att fylla Nollans fritid.
        \item[att] ordna nolluppdrag.
        \item[att] framställa F-nollmodulen.
        \item[att] utvärdera mottagningen.
        \item[att] tillse att alla Nollan, om de så önskar, bär nollbricka.
        \item[att] söka sponsring för mottagningen.
        \item[att] tillse att valen Åke bär nollbricka om gamble så tycker.
    \end{description}
    
    \subsection{Övrigt}
    \begin{description}
        \item[att] i samråd med Djungelpatrullen och F6 tillse att sektionsmärket i Olgas trappor målas i samband med mottagningen.
        \item[att] ordna en trevlig sammankomst med andra mottagningskommittéer.
    \end{description}
    
    \subsection{Valberedning}
    \aliggvalber{}
    
    \section{Postspecifika uppdrag}
    \subsection{Ordförande}
    FnollK:s ordförande är tillsammans med FnollK:s kassör ansvarig för kommitténs ekonomi. Ordförande skall leda FnollK:s arbete samt vara en kontaktlänk mellan FnollK och övriga kommittéer, nämnder samt sektionsstyrelsen. 
    \begin{itemize}
        \item FnollKs ordförande är ledamot i sektionsstyrelsen.
        \item FnollKs ordförande skall vara sektionens representant i Chalmers studentkårs samarbetsorgan för mottagningen, MOS.
    \end{itemize}
    Det åligger FnollKs ordförande:
    \begin{description}
        \item[att] tillse att kommitténs åligganden utförs.
        \item[att] ansvara för att god kontakt hålls med andra mottagningskommittéer.
    \end{description}
    
    \subsection{Vice ordförande}
    Det åligger FnollK:s vice ordförande:
    \begin{description}
        \item[att] vara kommitténs suppleant i sektionsstyrelsen.
        \item[att] bistå ordföranden i dennes uppgifter så att de utförs på bästa sätt.
        \item[att] i ordförandens frånvara utföra dennes uppdrag.
    \end{description}
    
    \subsection{Kassör}
    \subsubsection{Allmänt}
    Kassören är tillsammans med ordföranden ansvarig för FnollK:s ekonomi.
    
    \subsubsection{Åligganden}
    Det åligger kassören i varje sektionskommitté:
    \begin{description}
        \item[att] kontinuerligt föra en granskningsbar redovisning gällande kommitténs ekonomi.
        \item[att] mot revisorerna kontinuerligt redovisa för den ekonomiska situationen.
    \end{description}
    
    \subsection{Phadderchef}
    Det åligger FnollK:s Phadderchef att tillse att Nollans behov av phaddrar tillgodoses. Phaddergruppschefen skall därtill nominera ett erforderligt antal phaddergruppsansvariga till sektionsstyrelsen.
    
    \subsection{Övriga ledamöter}
    FnollK:s övriga ledamöter skall hjälpa FnollK:s ordförande i dennes uppgifter så att de utförs på bästa sätt.
\end{foreningenv}

\end{document}