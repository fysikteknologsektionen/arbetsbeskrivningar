\documentclass[a4paper]{article}
\usepackage{preamble}
\begin{document}

\renewcommand{\forening}{Dumvästinnehavare} % Används för headern bl.a.

\begin{foreningenv}{\forening{}} % Hanterar mycket av logiken kring enskild/gemensam kompilering. Argumentet blir rubriken
    \section{Allmänt}
    \begin{itemize}
    \item Dumvästinnehavaren är den som av sektionen utnämnts till dum.
    \item Det ska väljas en Dumvästinnehavare på varje sektionsmöte.
    \item Dumvästinnehavaren har funktionärsstatus på sektionen.
    \end{itemize}
    
    \section{Generella åligganden}
    \aliggsektfunkt{}
    
    \section{Specifika åligganden}
    Det åligger Dumvästinnehavaren:
    \begin{description}
      \item[att] bära Dumvästen vid varje av Fysikteknologsektionen anordnade aktiviteter.
      \item[att] tillse att Dumvästinnehavarens namn listas i dokumentet \textit{Förteckning över dumvästinnehavare} nästföljande sektionsmöte.
    \end{description}
    
    \section{Reglemente angående Dumvästens utdelande}
    \begin{itemize}
      \item Den, som med avsikt att erhålla Dumvästen utfört dumheter
      bör ej vara kvalificerad till denna.
      \item Kriminella handlingar är ej kvalificerade till Dumvästen,
      såvida inte sektionsmötet anser detta.
      \item Om ej tillräckligt kvalificerad dumhet nomineras, kvarstår Dumvästen hos innehavaren för tillfället.
    \end{itemize}
    
    Förteckning över Dumvästinnehavare genom tiderna tillhandahålls av sektionsstyrelsen.
    
    \vspace{3em}
    
    \textbf{Förteckningen över dumvästinnehavare har flyttats till \href{https://ftek.se/dumvast/}{https://ftek.se/dumvast/}} % Ta bort vid årsskiftet 2020/2021
\end{foreningenv}

\end{document}