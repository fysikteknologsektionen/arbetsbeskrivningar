\documentclass[a4paper]{article}
\usepackage{preamble}
\begin{document}

\renewcommand{\forening}{Spidera} % Används för headern bl.a.

\begin{foreningenv}{\forening{}} % Hanterar mycket av logiken kring enskild/gemensam kompilering. Argumentet blir rubriken
    \section{Allmänt}
    \begin{itemize}
        \item Spidera har funktionärsstatus på sektionen.
        \item Spideras syfte är att administrera och utveckla sektionens internetportal.
    \end{itemize}
    Spidera består av en nätmästare samt 1--9 nätmakare. 
    
    \section{Generella åligganden}
    \aliggsektfunkt{}
    
    \section{Specifika åligganden}
    Det åligger \forening{}:
    \begin{description}
        \item[att] administrera och utveckla Fysikteknologsektionens internetportal.
    \end{description}
    
    \section{Postspecifika åligganden}
    \subsection{Nätmästare}
    Det åligger Spideras nätmästare:
    \begin{description}
          \item[att] driva arbetet inom Spidera.
          %\item[att] fungera som länk mellan sektionsstyrelsen och Spidera. nyttnytt
          \item[att] kalla till möten med Spidera.
    \end{description}
    
    \subsection{Nätmakare}
    Det åligger Spideras nätmakare:
    \begin{description}
      \item[att] vara Nätmästaren behjälplig.
    \end{description}
\end{foreningenv}

\end{document}