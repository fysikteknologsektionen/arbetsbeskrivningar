

\renewcommand{\dateseparator}{-} %Omdefinierar hur datum visas

\renewcommand{\forening}{FARM}

\begin{center}
\LARGE{\textbf{Arbetsordning \forening}}
\end{center}
%\vspace{1cm}





\section{Allmänt}
\begin{itemize}
\item \forening \ är sektionens arbetsmarknadsgrupp. 
\item \forening \ har kommittéstatus på sektionen.
\item \forening:s syfte är att dels informera företag om programmen Teknisk Fysik samt Teknisk matematik med tillhörande mastersprogram och dess fördelar. \forening \ skall därutöver även tillse att sektionens medlemmar får information om deras möjligheter i arbetslivet efter slutförda studier.\
\item \forening \ skall ansvara för kommunikationen mellan föreningar och kommittéer på sektionen och alumniföreningen Forum för Tekniska Fysiker samt samordna gemensamma evenemang.\\
\end{itemize}

\forening \ består av följande poster:
\begin{itemize}
\item Ordförande
\item Vice ordförande
\item Kassör
\item 0--5 ledamöter
\end{itemize}

\forening \ har 0 övriga medlemmar. 

\section{Generella åligganden}

Det åligger varje sektionskommitté att:
\begin{description}
    \item[att] på det ordinarie sektionsmöte som följer på det, då kommittén blivit invald, presentera en verksamhetsplan för det kommande verksamhetsåret.
      \item[att] tillsammans med Fysikteknologsektionens valberedning lägga fram förslag på efterträdare till respektive sektionskommitté.
      \item[att] inför inval av poster och förtroendeposter i respektive sektionskommitté, tillse att samtliga sektionens medlemmar har givits fullgod möjlighet att med lätthet införskaffa information och bekantskap med respektive sektionskommitté och dess arbetsuppgifter.
      \item[att] kontinuerligt dokumentera sin verksamhet, som stöd till näst\-komm\-ande års kommitté.
      \item[att] delta i de av Djungelpatrullen anordnade städdagarna två gånger per
      läsår. Om detta uppfylls får de närvarande gå gratis på nästa
      sektionsaktivafest.
      \item[att] under inga omständigheter vid med sektionen associerat arrangemang eller vid med sektionen associerad festlighet servera, försälja eller bjuda på öhl av märket Pripps, förutom vid häfv eller arrangemang i kårhuset.
      
    \end{description}

\section{Specifika åligganden}
Det åligger \forening \ att:
\begin{description}
\item[att] arrangera studiebesök och branschkvällar.

\item[att] anordna sektionens arbetsmarknadsdag.

\item[att] hålla ordning på platsannonser och förslag till examensarbeten som inkommer till sektionen och anslå dessa i sektionslokalen.

\end{description}

\subsection{Valberedning}
\subsubsection{Förberedelse}
Inför valberedningen är kommittén skyldig att bilda sig en detaljerad uppfattning om varje aspirant till föreningen som skall valberedas.\\
Därtill och kommittén skyldig att i samråd med valberedningen upprätta en kravprofil för kommittén i helhet samt för kommitténs förtroendeposter.

\subsubsection{Intervjuer}
Under valberedningens intervjuer skall \forening \ utse två representanter att deltaga under dessa samt under de efterföljande samtalen i syfte att nominera kandidater.\\
De representanter som utses får själv inte valberedas för eller söka någon post i denna förening.

\section{Postspecifika uppdrag}

\subsection{Ordförande}
\forening s ordförande är tillsammans med \forening s kassör ansvarig för kommitténs ekonomi. Ordförande skall leda \forening s arbete samt vara en kontaktlänk mellan \forening \ och övriga kommittéer, nämnder samt sektionsstyrelsen. Det åligger \forening s ordförande att tillse att kommitténs åligganden utförs.

\begin{itemize}
\item  \forening s ordförande är ledamot i sektionsstyrelsen.
\end{itemize}

\subsection{vice Ordförande}
Det åligger vice ordföranden:
\begin{description}
\item[att] vara kommitténs suppleant i sektionsstyrelsen. 
\item[att] bistå ordföranden i dennes uppgifter så att de utförs på bästa sätt.
\item[att] i ordförandens frånvaro utföra dennes uppdrag.
\end{description} 

\subsection{Kassör}
Kassören är tillsammans med ordföranden ansvarig för \forening s ekonomi. Det åligger kassören att kontinuerligt föra en granskningsbar redovisning gällande \forening s ekonomi.\\

\subsubsection{Det åligger kassören i varje sektionskommitté:}
\begin{description}
\item[att] mot revisorerna kontinuerligt redovisa för den ekonomiska situationen.

\end{description}


\subsection{Övriga ledamöter}
\forening s övriga ledamöter skall hjälpa \forening s ordförande i dennes uppgifter så att de utförs på bästa sätt.

\section{Tolkning och ändring}
Tolkning av denna arbetsordning görs av sektionsstyrelsen.\\ Då Fysikteknologsektionens Stadga, Reglemente eller policyer motsäger denna arbetsordning har Stadga och Reglemente företräde. Då kårens Stadga, Reglemente eller policyer motsäger denna arbetsordning har kårens Stadga, Reglemente och policyer företräde.\\
Ändring och tillägg av denna arbetsordning görs av sektionsstyrelsen. Ändring skall redovisas vid nästkommande sektionsmöte. 

Första versionen av denna arbetsordning antogs av sektionsmötet läsperiod 1 verksamhetsåret 14/15.

\newpage
