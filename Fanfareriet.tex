\renewcommand{\dateseparator}{-} %Omdefinierar hur datum visas

\renewcommand{\forening}{Fysikteknologsektionens Fanfareri}

\begin{center}
\LARGE{\textbf{Arbetsordning Fanfareriet}}
\end{center}
%\vspace{1cm}


\section{Allmänt}
\begin{itemize}
\item Fanfareriet har funktionärsstatus på sektionen.
\item Fanfareriet består av en flaggmarskalk och 1-2 fanbärare.
\item Fanfareriets syfte är att ta hand om sektionens fanor och flaggor
\item Dragos är Fanfareriets högsta befäl.
\end{itemize}

Fanfareriet består av en flaggmarskalk och 1-2 fanbärare.

\section{Åligganden}
\subsection{Det åligger varje sektionsfunktionär:}
    \begin{description}
      \item[att] delta i de av Djungelpatrullen anordnade städdagarna två gånger per
      läsår då sektionslokalen och sektionens motorfordon städas. Om detta uppfylls får de närvarande gå gratis på nästa
      sektionsaktivafest.
    \end{description}

\subsection{Det åligger Fanfareriet:}
    \begin{description}
      \item[att] sköta sektionens fanor och flaggor.
    \end{description}

\section{Postspecifika åligganden}
\subsection{Flaggmarskalken}
Det åligger flaggmarskalken:
    \begin{description}
      \item[att] hissa Fanstyget i sektionens flaggstång:
	\begin{itemize}
	  \item Dag då sektionsmöte hålles.
	  \item Dag då nollkamp utkämpas.
	  \item På Nollans uppropsdag.
	  \item På sektionens dag.
	  \item På Emil och Emilias namnsdag den 14 november.
	  \item På Lee Falks dödsdag den 13 mars (halv stång).
	  \item Annan dag av synnerlig vikt för sektionen.
	\end{itemize}
      \item[att] hissa svensk flagga i sektionens flaggstång på allmän
      flaggdag, annan dag av synnerlig vikt för Chalmers eller
      nationen samt att härvid tillse att officiella flaggningsregler
      i största möjliga mån tillämpas.
      \item[att] vid tillfällen enligt eget gott omdöme, hissa
      sektionens eventuella övriga flaggor i sektionens flaggstång.
      \item[att] vid ovannämnda tillfällen informera sektionens
      medlemmar om anledningen till att Fanstyget, svensk flagga eller
      annan flagga är hissad.
      \item[att] meddela styrelsen skador eller andra brister hos
      Fanstyget, sektionens svenska flagga, sektionens flaggstång
      eller sektionens eventuella övriga flaggor.
      \item[att] vid förfall utse ersättare att utföra flaggmarskalkens åligganden.
    \end{description}

\subsection{Fanbärarna}
Det åligger Fanbärarna:
    \begin{description}
      \item[att] när så krävs bära Fysikteknologsektionens fana, iklädd
      högtidsdräkt, vid:
        \begin{itemize}
	  \item Valborg.
	  \item Mottagningens första dag.
	  \item Mösspåtagningen 1 oktober.
	  \item Andra tillfällen av vikt när sektionens fana ska bäras.
	\end{itemize}
    \end{description}




\section{Tolkning och ändring}
Tolkning av denna arbetsordning görs av sektionsstyrelsen.\\ Då Fysikteknologsektionens Stadga, Reglemente eller policyer motsäger denna arbetsordning har Stadga och Reglemente företräde. Då kårens Stadga, Reglemente eller policyer motsäger denna arbetsordning har kårens Stadga, Reglemente och policyer företräde.\\
Ändring och tillägg av denna arbetsordning görs av sektionsstyrelsen. Ändring skall redovisas vid nästkommande sektionsmöte. 
\newpage