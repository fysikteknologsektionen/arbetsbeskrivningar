\renewcommand{\dateseparator}{-} %Omdefinierar hur datum visas

\renewcommand{\forening}{Balnågonting }

\begin{center}
\LARGE{\textbf{Arbetsordning \forening}}
\end{center}
%\vspace{1cm}




\section{Allmänt}
\begin{itemize}
\item \forening \ är sektionens Balnågonting
\item Balnågonting har funktionärsstatus på sektionen. 
\item Balnågontings syfte är att anordna bal, middag samt kringaktiviteter som tillhör balen.
\end{itemize}


Balnågonting har 0-5 medlemmar samt en representant vardera från FnollK, F6 och Djungelpatrullen.


\section{Åligganden}
\subsection{Det åligger varje sektionsfunktionär:}
    \begin{description}
      \item[att] delta i de av Djungelpatrullen anordnade städdagarna två gånger per
      läsår då sektionslokalen och sektionens motorfordon städas. Om detta uppfylls får de närvarande gå gratis på nästa
      sektionsaktivafest.
    \end{description}
\subsection{Det åligger Balnågonting:}
		\begin{description}
\item[att] under mottagningen anordna en
  			bal, öppen för alla F- och \-TM-tekno\-log\-er, men främst riktad mot Nollan och phaddrar.
  		\item[att] se till att den anordnade balen är av god kvalitet och ordning, 
  			samt håller den klass som kan anses värdig en bal.
		\item[att] anordna sektionens middag.
		\item[att] vid inval av nya Balnågonting överräcka en höbal som symbolisk gest.
  	\end{description}


\section{Tolkning och ändring}
Tolkning av denna arbetsordning görs av sektionsstyrelsen.\\ Då Fysikteknologsektionens Stadga, Reglemente eller policyer motsäger denna arbetsordning har Stadga och Reglemente företräde. Då kårens Stadga, Reglemente eller policyer motsäger denna arbetsordning har kårens Stadga, Reglemente och policyer företräde.\\
Ändring och tillägg av denna arbetsordning görs av sektionsstyrelsen. Ändring skall redovisas vid nästkommande sektionsmöte. 

\newpage

