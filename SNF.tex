

\renewcommand{\dateseparator}{-} %Omdefinierar hur datum visas

\renewcommand{\forening}{SNF}

\begin{center}
\LARGE{\textbf{Arbetsordning SNF}}
\end{center}
%\vspace{1cm}





\section{Allmänt}
\begin{itemize}
\item \forening \ är sektionens Studienämnd.\
\item \forening \ har nämndstatus på sektionen.
\item \forening:s syfte är att sköta studiebevakningen på sektionen genom att kontinuerligt granska kurser som ger på programmen Teknisk Fysik, Teknisk Matematik samt associerade masterprogram.
\end{itemize}

\forening \ består av förljande poster:
    \begin{itemize}
        \item Ordförande
        \item Vice ordförande
        \item Kassör
        \item Sekreterare
        \item Kandidatansvarig
        \item Masteransvarig
        \item Veckobladerist
        \item Årskursrepresentant åk. 1
        \item Ledamot
    \end{itemize}

Där ordförande, vice ordförande samt kassör anses vara förtroendeposter. \\

Alla förtroendeposter nomineras av valberedningen, och övriga poster väljes av sektionsmötet enligt reglemente. Om någon av av övriga poster ej tillsätts enligt ovan kan de fyllnadsväljas av sektionsstyrelsen, men detta skall fastslås på påföljande sektionsmöte.

\section{Generella åligganden}
Det åligger \forening:
    \begin{description}
      \item[att] godkänna kursutvärderare.
      \item[att] sammanträda minst tre gånger per läsperiod.
      \item[att] informera F-och TM-teknologen i frågor rörande respektive utbildning.
      \item[att] ansvara för utvecklandet av utbildningsbevakningen på
				F-tekno\-log\-sektionen.
      \item[att] inför Fysikteknologsektionen svara för att F- och TM-teknologernas
      intressen i studie\-frågor och studiemiljö bevakas på ett
      tillfredsställande sätt.
      \item[att] på verksamhetsårets första sektionsmöte presentera en verksamhetsplan för det kommande läsåret.
      \item[att] ordna arrangemang i studiebefrämjande syfte.
      \item[att] tillsammans med Fysikteknologsektionens valberedning lägga fram förslag på efterträdare till nämnden.
      \item[att] inför inval av poster och förtroendeposter i nämnden, tillse att samtliga sektionens medlemmar har givits fullgod möjlighet att med lätthet införskaffa information och bekantskap med nämnd och dess arbetsuppgifter.
      \item[att] representera Fysikteknologsektionen i F:s och TM:s programråd.
      \item[att] delta i de av Djungelpatrullen anordnade städdagarna två gånger per
      läsår. Om detta uppfylls får de närvarande gå gratis på nästa
      sektionsaktivafest.
      \item[att] under inga omständigheter vid med sektionen associerade arrangemang eller med sektionen associera festlighet servera, försälja eller bjuda på öhl av märket Pripps, förutom vid häfv eller arrangemang i kårhuset.
    \end{description}


\subsection{Övrigt}
 Därtill åligger det \forening :
\begin{description}
    \item[att] i samband med Chalmers studentkårs pubrunda anordna Cocktailparty i varje läsperiod.
\end{description}

\subsection{Valberedning}
\subsubsection{Förberedelse}
Inför valberedningen är studienämnden skyldig att bilda sig en detaljerad uppfattning om varje aspirant till föreningen som skall valberedas.\\
Därtill är kommittén skyldig att i samråd med valberedningen upprätta en kravprofil för kommittén i sin helhet samt för kommitténs förtroendeposter.

\subsubsection{Intervjuer}
Under valberedningens intervjuer skall \forening \ utse två representanter att deltaga under dessa samt under de efterföljande samtalen i syfte att nominera kandidater.\\
De representanter som utses får själv inte valberedas för eller söka någon post i denna förening.

\section{Postspecifika uppdrag}

\subsection{Ordförande}
\forening s ordförande är tillsammans med \forening:s kassör ansvarig för kommitténs ekonomi. Ordförande skall leda \forening:s arbete samt vara en kontaktlänk mellan \forening \ och övriga kommittéer, nämnder samt sektionsstyrelsen. Det åligger \forening:s ordförande att tillse att nämndens åliggande utförs.\\

\begin{itemize}
\item \forening s ordförande är ledamot i sektionsstyrelsen.
\end{itemize}

Det åligger \forening :s innevarande ordförande:
\begin{description}
    \item[att] tillse att studienämndens åligganden utförs. 
    \item[att] leda studienämndens verksamhet.
    \item[att] kalla studienämnden till sammanträde.
    \item[att] handha studienämndens handlingar.
    \item[att] underteckna studienämndens handlingar.
    \item[att] i studie- och studiemiljöfrågor representera Fysikteknologsektionen och föra dess talan.
    \item[att] representera F och TM i Utbildningsutskottet, UU, och vid förhinder tillse att suppleant deltager.
\end{description}
    
    
\subsection{Vice ordförande}
Det åligger vice ordföranden:

\begin{description}
      \item[att] assistera ordföranden i dennes åligganden.
      \item[att] ersätta ordföranden när denne inte är närvarande. 
      \item[att] vara studienämndens suppleant i sektionsstyrelsen.
      \item[att] ansvara för studiesociala evenemang.
      \item[att] ansvara för av \forening\  arrangerade räkneövningstillfällen.
\end{description}


\subsection{Kassör}
Kassören är tillsammans med ordförande ansvarig för \forening \ ekonomi. Det åligger kassören att kontinuerligt föra en granskningsbar redovisning gällande \forening \ ekonomi.\\

Det åligger kassören att:
\begin{description}
    \item[att] mot revisorerna kontinuerligt redovisa för den ekonomiska situationen.
    \item[att] sköta och ansvara för studienämndens ekonomi tillsammans med ordförande.
    \item[att] kontinuerligt föra en granskningsbar redovisning gällande studienämdens ekonomi.
\end{description}

\subsection{Kandidatansvarig}
Det åligger \forening:s kandidatansvarig:
\begin{description}
      \item[att] ansvara för att kursutvärderingsansvariga tillsätts.
\end{description}

\subsection{Veckobladerist}
  Det åligger veckobladeristen (VBL):
\begin{description}
      \item[att] hålla Veckobladeriets hemsida uppdaterad.
      \item[att] komplettera Veckobladeriets arkiv, så att de är aktuella.
      \item[att] vårda minnet av det enorma arbetet de gamla Veckobladeristerna har genomfört genom att en gång per år anordna en omsits åt dem. 
\end{description}

\subsection{Sekreterare}
Det åligger \forening:s sekreteraren:
\begin{description}
    \item[att] tillse att protokoll förs på studienämndens möten.
    \item[att] anslå nämndens protokoll senast efter 2 läsveckor via sektionens officiella kommunikationskanaler. 
\end{description}

\subsection{\forening :s årskursrepresentant}
    Det åligger \forening:s årskursrepresentant:
\begin{description}
      \item[att] speciellt bevaka studiefrågor och studiesociala miljö i första årskursen.
      \item[att] i första årskursen informera om studienämndens verksamhet.
      \item[att] anordna övningstentor för förstaårsstudenterna på F och TM under första.
\end{description}

\subsection{Mastersansvarig}
Det åligger \forening:s mastersansvarig:
\begin{description}
      \item [att] handha studiebevakningen på masterprogram
      associerade med programmet Teknisk Fysik och/eller programmet Teknisk Matematik.
      \item [att] representera studienämnden på masterprogram associerade med
      programmet Teknisk Fysik och/eller programmet Teknisk Matematik.
      \item [att] representera studienämnden i Fysikteknologsektionens Masterenhet.
\end{description}
    

\subsection{\forening :s medlemmar}
\forening:s medlemmar och ledamöter skall hjälpa \forening:s ordförande i dennes uppgifter så att de utförs på bästa sätt.

Det åligger \forening :s medlemmar
\begin{description}
      \item[att] hjälpa VBL att komplettera Veckobladeriets arkiv.
      \item[att] i frånvaro av VBL utföra VBL:s uppgifter.
      \item[att] vara ordföranden behjälplig.
\end{description}
    

\section{Tolkning och ändring}
Tolkning av denna arbetsordning görs av sektionsstyrelsen.\\ Då Fysikteknologsektionens Stadga, Reglemente eller policyer motsäger denna arbetsordning har Stadga och Reglemente företräde. Då kårens Stadga, Reglemente eller policyer motsäger denna arbetsordning har kårens Stadga, Reglemente och policyer företräde.\\
Ändring och tillägg av denna arbetsordning görs av sektionsstyrelsen. Ändring skall redovisas vid nästkommande sektionsmöte. 
\newpage